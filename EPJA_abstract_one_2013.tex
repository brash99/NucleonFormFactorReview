\documentclass[12pt]{article}

\textwidth=163mm
\textheight=237mm

\setlength{\voffset}{-20mm}
\oddsidemargin -5mm
\evensidemargin -5mm

\usepackage{epsf}
\usepackage[dvips]{graphicx}

\begin{document}


\begin{center}

{\bfseries Nucleon Form Factors}

\vspace{10pt}

Vina Punjabi$^1$, C.F. Perdrisat$^2$, M.K. Jones$^3$, E.J. Brash$^4$

\vspace{10pt}

$^1${Physics Department, Norfolk State University, Norfolk,Virginia, USA}\\
$^2${Physics Department, The College of William and Mary, Williamsburg, Virginia, USA}\\ 
$^3${Thomas Jefferson National Accelerator Facility, Newport News, Virginia, USA}\\
$^4${Physics Department, Christopher Newport University, Newport News, Virginia, USA}\\ 

% Enter contact e-mail address here.

%\centerline{Contact email: {\it punjabi@jlab.org}} 

\vspace{18pt} % Do not modify

\end{center}
%%%
%%% Abstract proper starts here.
%%%
Precise proton and neutron form factor measurements at Jefferson Lab using double 
polarization method have made tremendous contribution to unravel the internal structure 
of the nucleon. The goal of accurate experimental measurements of the nucleon form factors is to
understand how their static properties and dynamical behavior emerge from QCD, the 
theory of the strong interactions between quarks. After the publication of Jefferson Lab 
proton form factor ratio data, there has been an enormous theoretical progress to reevaluate the
new picture of the proton. We will review the experimental and theoretical progress, including 
the progress in generalized parton distributions and flavor separation, Dyson-Schwinger 
equations (DSEs) calculations, and significant improvements made in a lattice QCD calculations. 


 
%%%
%%% End of abstract.
%%%
\end{document}






