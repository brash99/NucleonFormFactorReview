\subsection{Electron Scattering Cross Section Experiments}
\label{subsec:xsection}

A recent experiment at the Mainz Microtron measured precision elastic cross-sections over a range of Q$^2$~=~0.003 to
1.0~GeV$^2$. 
\subsubsection{G$^M_p$ - Dipole Form Factor Parametrization}
\label{subsubsec:gmp}

\subsubsection{G$^E_p$ - Elastic Scattering Results}
\label{subsubsec:gep}

\subsubsection{SuperRosenbluth Separation Results}
\label{subsubsec:superrosen}
 Taking advantage of the high duty  factor of modern accelerators, cross-sections can be 
 measured to high precision over a range of $\epsilon$ in a relative short time period.
 Instead of detecting the elastically scattered electron, $(e,e^{\prime})$, an experiment 
 which detected the elastically scattered proton to
identify elastic reactions,  $(e,p)$, was run at Jefferson Lab in 2002. 
The same experimental approach of extraction the form factors by measuring elastic cross-sections
at fixed Q$^2$ and different $\epsilon$ by varying the beam energy is used. The experimental method takes advantage of the fact
that the proton momentum is constant for all epsilon at a fixed Q$^2$. 
In addition for  $(e,p)$ , the detected proton rate and the radiation corrections 
have a smaller dependence on $\epsilon$ compared to  $(e,e^{\prime})$ experiments. 
All this combines to  reduce the $\epsilon$ dependent systematic error compared to $(e,e^{\prime})$.
The form factors were measured at Q$^2$ = 2.64, 3.10 and 4.60 GeV$^2$.
In the Figure, the measurements of $\mu$G$^E_p$/G$^M_p$ from the $(e,p)$ reactions are plotted compared to a global analysis of previous
measurements from $(e,e^{\prime})$ experiments and the agreement between the different methods is excellent. 
The plotted errors for the $(e,p)$ data set  are the combined statistical and systematic
error and the error on the extraction of $\mu$G$^E_p$/G$^M_p$ has been improved by factors of two to three. The detection
of scattered electrons or the scattered protons experiment techniques has different systematics, so the agreement between
the two techniques indicate that experimental systematic error are understood.

With the success of the first Jefferson Lab $(e,p)$ experiment, a subsequent experiment was run at Jefferson Lab in Hall C.
The experiment measured cross-sections at a total of 102 kinematic settings covering  a wide Q$^2$ range from 0.4 to 5.76~GeV$^2$
with at least three $\epsilon$ points per Q$^2$. The emphasis was at each Q$^2$ to measure as wide an $\epsilon$ range as possible.
At Q$^2$~=~1~GeV$^2$, thirteen $\epsilon$ points were measured ranging from $\epsilon$~=0.05 to 0.98, with eight of the points
above $\epsilon$ = 0.8. Similarly for At Q$^2$~=~2.3~GeV$^2$, ten $\epsilon$ points were measured ranging from $\epsilon$~=0.07 to 0.92, 
with five of the points above $\epsilon$ = 0.7.  The wide range of  $\epsilon$ at a fixed Q$^2$ allows a check of the non-linearity 
in the $\epsilon$  dependence of the cross-section which would be a sign of two-photon exchange reactions effecting the cross-sections.
The effects from two-photon exchange could have a dramatic  $\epsilon$  dependence near  $\epsilon$~=~1.


\subsubsection{G$^M_n$/G$^E_n$ - QuasiElastic Scattering Results}
\label{subsubsec:gmn}
