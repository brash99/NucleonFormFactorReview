The two Sachs form factors, $G_{Ep}$ and $G_{Mp}$, required 
to describe the nucleon charge- and magnetization distribution have been 
traditionally obtained by Rosenbluth separation \cite{rosen}. The $G_{Mp}$-data 
obtained by this method have shown good internal consistency
up to 30 GeV$^2$. However, the characterization of $G_{Ep}$ has suffered from 
large inconsistencies in the data base, which are now understood to be the 
result of the fast decrease of the contribution of $G_{Ep}$ to the cross section.
Although the structure of the proton has been taken for well 
known until recently, the experimental results from three GEp experiments from JLab
showed that it held secrets which are only now being revealed.      

The structure of the nucleons has been investigated experimentally with rigour  
over last 60 years using elastic electron scattering. 
The recent generation of electron accelerators with high polarization and high current electron beams, at MIT-Bates, the Mainz
Microtron (MAMI), and the Continuous Electron Beam Accelerator Facility (CEBAF)
of the Jefferson Lab (JLab), have made it possible to investigate the internal structure of the nucleon with 
extreme precision. In particular, the new generation of experiments that measures polarization variables like
target asymmetry and recoil-polarization have
allowed one to obtain the proton and neutron electromagnetic form factors accurately to large four momentum trafsfer, $Q^2$.
In this section we describe the form factors obtained from cross sections and double polarization experiments for 
proton and neutron.
